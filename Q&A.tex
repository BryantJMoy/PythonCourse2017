QUESTIONS THAT EMERGED DURING THE PYTHON COURSE 2017

- Is there a way of knowing or seeing all the objects that exist in the global environment?
A= Yes! And even inside the local environment: globals(), locals() and dir()

- Is there a debugging function/library in Python, similar to debug() in R?
A= Not identical but with a super similar logic. The library is called "pdb" (run >> pip install pdb in the terminal first) and then import it in a script you want to test. A good tutorial of how to use it can be found here: https://pythonconquerstheuniverse.wordpress.com/2009/09/10/debugging-in-python/

- Some useful functions to keep on your radars:
	* map(): applies a function to all the items in an input_list
	* filter(): creates a list of elements for which a function returns true
	* enumerate(): allows us to loop over something and have an automatic counter
	* reduce(): it applies a rolling computation to sequential pairs of values in a list.

- How to schedule Python scripts? Use CRON!
	1. Type in terminal: nano <name of cron script>.cron
	2. Syntax:
		* * * * * python <name of python script>.py
		1st * : minutes [0-60]
		2nd * : hours [0-23]
		3rd *: day of the month [1-31]
		4th *: month [1-12]
		5th *: day of the week [MON-SUN]
	3. SAVE IT [CTRL+X, Y, Enter]
	4. Load the cron document into the "crontab": crontab <name of cron script>.cron
	5. Check the document is there: crontab -l
	6. Voil�!
	


		
	
		
